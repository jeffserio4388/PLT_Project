\documentclass[./Report_main.tex]{subfiles}
\begin{comment}
If you want a box around your answer and that answer is an
equation then use \boxed{$$ equation $$} 

if you want to indent a block of text:
\begin{adjustwidth}{cm of right indent}{cm of left indent}
% paragraph to be indented
\end{adjustwidth}

if you just want one indent for one line 
use \indent per intended indent per line

A sections numbers automatically, so if the number of 
the problem is out of order it would be easier to 
just indent and bold the sections and subsections
and not use the \section{} kind of commands

\newpage makes a new page

$normal math mode$
$$Special math mode$$

to include an image use
\includegraphics{image_name}
image_name is the file name (.png) without the extension. The file
name cannot have any spaces or any periods other than the one before
the file extension.

To include a codeblock use
\begin{lstlisting}
ExampleCode(blah, blah)
{
	it does tabbing and everything;
	for (coloring of major languages like java){
		add the folloing to the \lstset tuple:
			language=<name_of_language>;
	}
}
\end{lstlisting}

\end{comment}


\begin{document}

%\tableofcontents

%\thispagestyle{empty}
%\newpage
% If you want to change how the subsubsection's are numbered
%\renewcommand{\thesubsection}{\thesection.\alph{subsection}.} 

%\setcounter{page}{0}
\chapter{Lessons Learned}
\section{Brandon}
I was surprised to learn how easy it is to build a very simply compiler and toy language. I though that might be the hardest part of the project - getting the scanner, parser, and codegen running. In fact, the most difficult part of the project was conceptually understanding the incredibly complex libuv library, deciding which features to incorporate in our language, and designing a plan to do that incrementally.\\\\
This fits in with my software industry experience (summer internships), in that usually the hardest part of a project is 1) conceptualizing the actual solution and 2) planning the incremental changes that add up to a significant impact. I've found if the planning is done right, implementing the incremental changes is not so hard (until it is).
%\subsection{Identifiers}
%\subsubsection{}
%\subsection{subsection}
%\subsubsection{subsubsection}
\end{document}

